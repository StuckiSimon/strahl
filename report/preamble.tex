%%% Local Variables:
%%% mode: latex
%%% TeX-master: "main"
%%% coding: utf-8
%%% End:
% !TEX TS-program = pdflatexmk
% !TEX encoding = UTF-8 Unicode
% !TEX root = main.tex
% Lorem Ipsum
\usepackage{lipsum}

% Language and Font
\usepackage[utf8]{inputenc}
\usepackage[T1]{fontenc}
\usepackage[ngerman, english]{babel}
\usepackage{lmodern}

% Mathematical Symbols
\usepackage{amssymb}

% Matrix etc.
\usepackage{amsmath}

% Images
\usepackage{graphicx} % for including graphics
\graphicspath{ {img/} } % root directory for graphics
\usepackage{subcaption} % figures inside figures (e.g. for side by side figures)
\usepackage{float} % required for positioning figures with [H] (and lots of other stuff)
\usepackage{wrapfig} % floating figures

% Import PDFs
\usepackage{pdfpages}


\usepackage[headsepline, footsepline, plainfootsepline]{scrlayer-scrpage}
\clearpairofpagestyles
\automark{chapter}
\ihead{\leftmark{}}
\ofoot[\thepage]{\thepage}

\setlength{\parindent}{0cm} % no indent for new paragraphs

% Finish paragraph so that it can be used as a title
\makeatletter
\renewcommand\paragraph{\@startsection{paragraph}{4}{\z@}%
            {-2.5ex\@plus -1ex \@minus -.25ex}%
            {1.25ex \@plus .25ex}%
            {\normalfont\normalsize\itshape\bfseries}}
\makeatother

% Package for adding todo notes and highlighting them in color
\usepackage[ngerman,linecolor=gray,bordercolor=gray,backgroundcolor=yellow]{todonotes}

% Code Highlighting
\usepackage{listings}
\definecolor{mygreen}{rgb}{0,0.6,0}
\definecolor{mymauve}{rgb}{0.58,0,0.82}
\lstset{ %
  basicstyle=\ttfamily,        % size of fonts used for the code
  breaklines=true,                 % automatic line breaking only at whitespace
  captionpos=b,                    % sets the caption-position to bottom
  commentstyle=\color{mygreen},    % comment style
  escapeinside={\%*}{*)},          % if you want to add LaTeX within your code
  keywordstyle=\color{blue},       % keyword style
  stringstyle=\color{mymauve},     % string literal style
  extendedchars=true,
  frame=single,
  frameround=tttt,
  framexleftmargin=1pt,
  framexrightmargin=1pt
}

% Hyperlinks
\usepackage[hidelinks]{hyperref}
\hypersetup{%
  pdftitle={Web-based Rendering of physical-based 3D Scenes using WebGPU},
% Use \plainauthor for final version.
  pdfauthor={Simon Stucki},
%  pdfauthor={\emptyauthor},
  pdfdisplaydoctitle=true, % For Accessibility
  bookmarksnumbered,
  pdfstartview={FitH}
}

% Enumerated list modification
\usepackage{enumitem}

% Diagrams
\usepackage{pgfplots}
\usepackage{pgfplotstable}
\usepackage{sfmath}
\usetikzlibrary{patterns}

% Removes the block justification from the bibliography and formats it more nicely than just \raggedright
\usepackage{ragged2e}

\newcommand{\pro}{\item[$+$]}
\newcommand{\con}{\item[$-$]}

% shortcut for emphasize
\newcommand{\e}{\emph}
\let\oldgls\gls
\let\oldGls\Gls
\let\oldglspl\glspl
\let\oldGlspl\Glspl
\renewcommand{\gls}[1]{{\e{\oldgls{#1}}}}
\renewcommand{\Gls}[1]{{\e{\oldGls{#1}}}}
\renewcommand{\glspl}[1]{{\e{\oldglspl{#1}}}}
\renewcommand{\Glspl}[1]{{\e{\oldGlspl{#1}}}}

% Shortcut for first glossary reference: lower-case & singular
\newcommand{\fgls}[2]{{
  \newglossaryentry{#1}
  {
    name={#1},
    description={{#2}}
  }\gls{#1}\footnote{#2}}}

% Shortcut for first glossary reference: lower-case & plural
\newcommand{\fglspl}[2]{{
  \newglossaryentry{#1}
  {
    name={#1},
    description={{#2}}
  }\glspl{#1}\footnote{#2}}}

% Shortcut for first glossary reference: Upper-case & singular
\newcommand{\fGls}[2]{{
  \newglossaryentry{#1}
  {
    name={#1},
    description={{#2}}
  }\Gls{#1}\footnote{#2}}}

% Shortcut for first glossary reference: Upper-case & plural
\newcommand{\fGlspl}[2]{{
  \newglossaryentry{#1}
  {
    name={#1},
    description={{#2}}
  }\Glspl{#1}\footnote{#2}}}

\pgfplotsset{compat=1.16}

\hyphenation{MaterialX}
\hyphenation{WebGPU}
\hyphenation{WebGL}
\hyphenation{OpenPBR}

% Code Syntaxhighlighting
% JavaScript
\definecolor{lightgray}{rgb}{0.95, 0.95, 0.95}
\definecolor{darkgray}{rgb}{0.4, 0.4, 0.4}
%\definecolor{purple}{rgb}{0.65, 0.12, 0.82}
\definecolor{editorGray}{rgb}{0.95, 0.95, 0.95}
\definecolor{editorOcher}{rgb}{1, 0.5, 0} % #FF7F00 -> rgb(239, 169, 0)
\definecolor{editorGreen}{rgb}{0, 0.5, 0} % #007C00 -> rgb(0, 124, 0)
\definecolor{brown}{rgb}{0.69,0.31,0.31}
\lstdefinelanguage{JavaScript}{
  morekeywords={typeof, new, true, false, catch, function, return, null, catch, switch, var, if, in, while, do, else, case, break, import, from, const},
  morecomment=[s]{/*}{*/},
  morecomment=[l]//,
  morestring=[b]",
  morestring=[b]'
}
\lstdefinestyle{JavaScript} {%
  % General design
%  backgroundcolor=\color{editorGray},
  basicstyle={\footnotesize\ttfamily},
  frame=b,
  % line-numbers
  xleftmargin={0.75cm},
  numbers=left,
  stepnumber=1,
  firstnumber=1,
  numberfirstline=true,
  % Code design
  identifierstyle=\color{black},
  keywordstyle=\color{blue}\bfseries,
  ndkeywordstyle=\color{editorGreen}\bfseries,
  stringstyle=\color{editorOcher}\ttfamily,
  commentstyle=\color{brown}\ttfamily,
  % Code
  language=JavaScript,
  alsodigit={.:;},
  tabsize=2,
  showtabs=false,
  showspaces=false,
  showstringspaces=false,
  extendedchars=true,
  breaklines=true,
}

% glsl
\lstdefinelanguage{glsl}{
  morekeywords={void, attribute},
  morecomment=[l]//,
}
\lstdefinestyle{glsl} {%
  % General design
  basicstyle={\footnotesize\ttfamily},
  frame=b,
  % line-numbers
  xleftmargin={0.75cm},
  numbers=left,
  stepnumber=1,
  firstnumber=1,
  numberfirstline=true,
  % Code design
  identifierstyle=\color{black},
  keywordstyle=\color{blue}\bfseries,
  ndkeywordstyle=\color{editorGreen}\bfseries,
  stringstyle=\color{editorOcher}\ttfamily,
  commentstyle=\color{brown}\ttfamily,
  % Code
  language=glsl,
  alsodigit={.:;},
  tabsize=2,
  showtabs=false,
  showspaces=false,
  showstringspaces=false,
  extendedchars=true,
  breaklines=true,
}

% wgsl
\lstdefinelanguage{wgsl}{
  morekeywords={void, attribute},
  morecomment=[l]//,
}
\lstdefinestyle{wgsl} {%
  % General design
  basicstyle={\footnotesize\ttfamily},
  frame=b,
  % line-numbers
  xleftmargin={0.75cm},
  numbers=left,
  stepnumber=1,
  firstnumber=1,
  numberfirstline=true,
  % Code design
  identifierstyle=\color{black},
  keywordstyle=\color{blue}\bfseries,
  ndkeywordstyle=\color{editorGreen}\bfseries,
  stringstyle=\color{editorOcher}\ttfamily,
  commentstyle=\color{brown}\ttfamily,
  % Code
  language=glsl,
  alsodigit={.:;},
  tabsize=2,
  showtabs=false,
  showspaces=false,
  showstringspaces=false,
  extendedchars=true,
  breaklines=true,
}

% json

\lstdefinelanguage{json}{
  basicstyle={\footnotesize\ttfamily},
  frame=b,
  % line-numbers
  xleftmargin={0.75cm},
  numbers=left,
  stepnumber=1,
  firstnumber=1,
  numberfirstline=true,
  tabsize=4,
  string=[s]{"}{"},
  stringstyle=\color{blue},
  comment=[l]{:},
  commentstyle=\color{black},
}

\lstdefinestyle{json} {%
  % Code
  language=json,
}

\makeatletter
\renewcommand\listoffigures{%
    \section{\listfigurename}% Used to be \section*{\listfigurename}
      \@mkboth{\MakeUppercase\listfigurename}%
              {\MakeUppercase\listfigurename}%
    \@starttoc{lof}%
    }
\makeatother

\makeatletter
\renewcommand\listoftables{%
    \section{\listtablename}% Used to be \section*{\listtablename}
      \@mkboth{\MakeUppercase\listtablename}%
              {\MakeUppercase\listtablename}%
    \@starttoc{lot}%
    }
\makeatother

\makeatletter
\renewenvironment{thebibliography}[1]
     {\section{\bibname}% <-- this line was changed from \chapter* to \section
      \@mkboth{\MakeUppercase\bibname}{\MakeUppercase\bibname}%
      \list{\@biblabel{\@arabic\c@enumiv}}%
           {\settowidth\labelwidth{\@biblabel{#1}}%
            \leftmargin\labelwidth
            \advance\leftmargin\labelsep
            \@openbib@code
            \usecounter{enumiv}%
            \let\p@enumiv\@empty
            \renewcommand\theenumiv{\@arabic\c@enumiv}}%
      \sloppy
      \clubpenalty4000
      \@clubpenalty \clubpenalty
      \widowpenalty4000%
      \sfcode`\.\@m}
     {\def\@noitemerr
       {\@latex@warning{Empty `thebibliography' environment}}%
      \endlist}
\makeatother