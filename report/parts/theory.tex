%%% Local Variables:
%%% mode: latex
%%% TeX-master: "../main"
%%% coding: utf-8
%%% End:
% !TEX TS-program = pdflatexmk
% !TEX encoding = UTF-8 Unicode
% !TEX root = ../main.tex

\section{Computer Graphics}

This section provides an overview of computer graphics, including its history, key concepts and applications.

Since the early days of computing, researchers have explored ways to process visual information using computers. While the field also encompasses aspects such as image processing or two-dimensional graphics, the main focus of this thesis is on three-dimensional computer graphics. While this includes topics such as animation, geometry processing, the main topics covered are geometry representation, rendering and shading.

\subsection{GPU}

The GPU is a specialized processor that is well-suited for workloads which require parallel processing. GPUs are widely used in computer graphics and other disciplines such as Machine Learning relying on parallel processing.

The first GPUs were developed in the 1990s and have since been integrated into most consumer hardware including smartphones, tablets, and laptops.

While CPU parallelization on consumer hardware is generally limited to a few cores, GPUs have thousands of compute units.

\subsection{Rendering Approaches}

\subsubsection{Rasterization}
Screen space ambient occlusion (SSAO)
RTAO
\subsubsection{Ray Tracing}

One of the earliest papers describing approaches to solve shadow casting was written as early 1968 by Appel \cite{appel1968shading}. The term global illumination was coined by Whitted in 1979 \cite{whitted2020OriginsOfGlobalIllumination}.

\todo{This is only for showing citation style}

\subsection{Graphic APIs}
\subsubsection{OpenGL}

OpenGL is an API for rendering 3d graphics. After its introduction in 1992, it was widely adopted. Subsequently, the standard has been ported to other platforms and has been extended with new features.

To date, OpenGL is still widely used in the industry, but it has been replaced by more modern APIs in recent years.

\subsubsection{WebGL}

WebGL is a graphics API based on OpenGL ES 2.0 for the web. It was initially released in 2011 and has since been adopted by all major browsers.

WebGL is designed to offer a rendering pipeline, but does not offer GPGPU capabilities. There have been efforts to extend with compute shaders, but efforts by the Khronos Group have been halted in favor of focusing on WebGPU instead.

\subsubsection{WebGPU}

WebGPU is a new web standard which is no longer based on OpenGL. One of the main capabilities is support for GPGPU by design. While all major browser vendors have announced intent to support WebGPU, to date only Chrome has shipped WebGPU for general use on desktop as well as mobile.
The standard is still in development and new features are being added.
Common 3d engines such as Babylon.js, Three.js, PlayCanvas and Unity have announced support for WebGPU.

\section{Ray Tracing}
\subsection{Monte Carlo Ray Tracing}
\subsection{Intersection Testing}
\fgls{BVH}{\e{Bounding Volume Hierarchy}, common tree-based acceleration structure}

\section{Physically Based Rendering}
\subsection{BxDF}
\subsection{Microfacet Theory}
\subsection{Exchange Formats}

In order to exchange 3d scenes between a multitude of applications, various standardized formats have been developed. These formats are optimized for different use cases depending on the requirements of the application.

\subsubsection{General Purpose Formats}

One of the most widely used formats is Wavefront OBJ which was established in the 1980s. The format is text-based and has basic support for materials and textures. However, it lacks support for more advanced features such animations. Additionally, due to its encoding, it is not well-suited for delivery over the web compared to more modern alternatives.

Formats such as the proprietary FBX by Autodesk established in 2006 can address the shortcomings in terms of advanced features. The format is supported by a wide range of applications.

\subsubsection{Interoperability Formats}

Other formats such as COLLADA have been developed to improve transporting 3d scenes between different applications. It is XML-based and has been established in 2004.

Since, other formats such as USD and has been open-sourced by Pixar in 2016.

\subsubsection{Runtime Formats}

For usage in end-user applications, such as for this thesis, the glTF format is well-suited. It has been established in 2015 and is designed to be efficient for transmission and loading of 3d scenes.

\subsection{Material Description}

\subsubsection{MaterialX}
\subsubsection{OpenPBR}
