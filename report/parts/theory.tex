%%% Local Variables:
%%% mode: latex
%%% TeX-master: "../main"
%%% coding: utf-8
%%% End:
% !TEX TS-program = pdflatexmk
% !TEX encoding = UTF-8 Unicode
% !TEX root = ../main.tex

\section{Computer Graphics}

This section provides an overview of computer graphics, including its history, key concepts and applications.

Since the early days of computing, researchers have explored ways to process visual information using computers. While the field also encompasses aspects such as image processing or two-dimensional graphics, the main focus of this thesis is on three-dimensional computer graphics. While this includes topics such as animation, geometry processing, the main topics covered are geometry representation, rendering and shading.

\subsection{GPU}

The GPU is a specialized processor that is well-suited for workloads which required parallel processing. GPUs are widely used in computer graphics and other disciplines relying on parallel processing such as Machine Learning.

\subsection{Rendering Approaches}

\subsubsection{Rasterization}
\subsubsection{Ray Tracing}

\subsection{Graphic APIs}
\subsubsection{OpenGL}

OpenGL is an API for rendering 3d graphics. After its introduction in 1992, it was widely adopted. Subsequently, the standard has been ported to other platforms and has been extended with new features.

To date, OpenGL is still widely used in the industry, but it has been replaced by more modern APIs in recent years.

\subsubsection{WebGL}
\subsubsection{WebGPU}

\section{Ray Tracing}
\subsection{Monte Carlo Ray Tracing}
\subsection{Intersection Testing}
BVH

\section{Physically Based Rendering}
\subsection{BxDF}
\subsection{Microfacet Theory}
\subsection{Exchange Formats}
\subsubsection{glTF}
\subsubsection{USD}

\subsubsection{MaterialX}
\subsubsection{OpenPBR}
