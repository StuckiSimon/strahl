%%% Local Variables:
%%% mode: latex
%%% TeX-master: "../main"
%%% coding: utf-8
%%% End:
% !TEX TS-program = pdflatexmk
% !TEX encoding = UTF-8 Unicode
% !TEX root = ../main.tex

This work introduces a web-based, open-source path tracer designed for rendering physically-based 3D scenes. It harnesses the capabilities of WebGPU and incorporates the OpenPBR surface shading model. While rasterization has been the prevailing real-time rendering technique on the web since the inception of WebGL in 2011, it exhibits limitations when aiming to achieve sophisticated lighting effects such as realistic shadows and reflections. This necessitates more complex techniques, often relying on pregenerated artifacts to attain the desired level of visual fidelity. Path tracing inherently addresses these weaknesses but at the expense of increased rendering times. This work focuses on specific use cases, such as industrial applications, where highly customizable products with significant variability are commonplace. In these contexts, real-time rendering performance is not an absolute requirement. Conversely, the enhanced realism offered by path tracing can be advantageous, eliminating the need for pregenerated assets typically employed with offline renderers. This work investigates the potential of WebGPU to facilitate path tracing on the web, enabling the integration of the OpenPBR standard for physically-based material representation. The outcome is a near real-time, open-source path tracer for rendering 3D scenes directly within a web browser.
