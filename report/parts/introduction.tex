%%% Local Variables:
%%% mode: latex
%%% TeX-master: "../main"
%%% coding: utf-8
%%% End:
% !TEX TS-program = pdflatexmk
% !TEX encoding = UTF-8 Unicode
% !TEX root = ../main.tex

Ray tracing is a powerful technique for rendering photorealistic sceneries. Historically, ray tracing has been mainly used in offline rendering due to its computational complexity. However, advancements in hardware and software have enabled real-time ray tracing in a variety of domains.
WebGPU is a new web standard that provides a robust API for leveraging the power of GPUs across different platforms. It's designed to be more efficient and flexible than its predecessor, WebGL. This makes it ideal to be used for real-time ray tracing in the browser.

\section{Use Cases}

The web is a versatile platform which can be used for a variety of applications. This section outlines some real-world use cases for web-based 3d rendering.

One such use case is e-commerce. Historically, companies with a vast portfolio of products have provided physical catalogues to their customers. This approach hit limits mainly for use cases where configurations of multiple components are possible and the number of combinations is vast or not enumerable. This is where real-time rendering can be used to provide a more interactive experience by showing the customer a realistic representation of the product instead of a list of static images.

As long as the configurations are enumerable it is theoretically possible to pre-render the combinations using offline renderers. However, this induces cost in terms of required processing power and storage.

When opting to use real-time rendering, one option to choose is remote rendering which employs a server to render the scene and stream the visualization to the browser. The main drawbacks of this approach include network latency and full reliance on network stability. Additionally, the cost of running the server falls on the service provider.

Another option is to use client-side rendering. Generally, rasterization approaches are used for real-time rendering. However, as will be discussed in the following sections, rasterization has limitations in terms of realism when having limited control over the models.

When all these requirements are considered, the need for a real-time ray tracing solution in the web becomes apparent.

\subsection{CAD Models}

The use of CAD models for designing and manufacturing processes is widespread in industry. These models often contain detailed information about the geometry as well as material properties. Such models are frequently used in mechanical engineering and product design.

Using such models as the basis for end user applications can be beneficial as it alleviates the need to create separate marketing models.

Some of the challenges when using CAD models is the lack of detailed material information. Another difficulty can be the complexity of the models. Frequently, the triangulated meshes are too fine-grained. There are approaches, such as algorithms used to generate level of detail (LOD) artifacts, to circumvent this issue.
Another important consideration are intellectual property (IP) rights when using CAD models from production processes. The models may contain proprietary information which should not be disclosed to the end user.

\subsection{Industry Use Cases}

One example of such a use case can be found at the company EAO. They produce industrial pushbuttons, keypads and other operator panels. The products are highly customizable and can be configured in a variety of ways. In order to facilitate the configuration process, a web-based configurator is optimal because it can be used on a large variety of devices without having to install additional software.

In addition to these factors, the pre-processing step should also define information on what kind of components can be added to the assembly and where they can be attached to. This can be done by providing meta information files containing the information. Alternatively, this can also be represented geometrically within the model by using identifiable shapes to attach components to.

\section{Prior Work}

\subsection{WebGPU}

Different other applications of WebGPU have been investigated in the past years. One such example is Dynamical.JS, a framework to visualize graphs \cite{dotson2022dynamicaljs}. Another example is RenderCore, a research-oriented rendering engine \cite{Bohak_Kovalskyi_Linev_Mrak_Tadel_Strban_Tadel_Yagil_2024}, or demonstrations on how to use WebGPU for client-side data aggregation \cite{kimmersdorfer2023webgpu}.

Investigation into the performance of WebGPU have been conducted and show that WebGPU can be faster than WebGL \cite{fransson2023performance, CHICKERUR2024919}.
 

\subsection{Web Path Tracers}

There are a variety of path tracers available for the web. Most of them are based on WebGL, a web standard which will be highlighted in the following sections.

The first experiments of using WebGL for path tracing were implemented as early as 2010. One such example is the demo by Evan Wallace showcasing a Cornell Box with basic primitive shapes such as spheres and planes.

Since then, a variety of open-source implementations for the web have been created.

\subsection{Three.js-based Ray Tracers}

Some of the most widely known path tracers are based on Three.js. Mainly two implementations are noteworthy: three-gpu-pathtracer by Garret Johnson and Three.js PathTracer by Erich Loftis.

https://github.com/gkjohnson/three-gpu-pathtracer
https://github.com/erichlof/THREE.js-PathTracing-Renderer
