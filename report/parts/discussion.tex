%%% Local Variables:
%%% mode: latex
%%% TeX-master: "../main"
%%% coding: utf-8
%%% End:
% !TEX TS-program = pdflatexmk
% !TEX encoding = UTF-8 Unicode
% !TEX root = ../main.tex

This section discusses the results in more detail. The experience gained by implementing a project using WebGPU is discussed. In addition, a more detailed discussion of the results is provided.

The ray tracer focuses on the given use case and is therefore not a general-purpose rendering engine. For example, it does not offer a physics engine, support for animations, or other features that are common in general-purpose rendering engines such as Three.js, Babylon.js or similar.

Generally, the ray tracing technique will be slower than rasterization-based approaches and is therefore not a silver bullet for all use cases.

\section{Findings}
\subsection{WebGPU Stability and Adoption}

Due to the current state of support for WebGPU in Safari and Mozilla Firefox, the production readiness of WebGPU is still limited. Safari has announced plans to support WebGPU and has launched a preview version \cite{SafariWebGPUSupport}. Firefox also has plans to support WebGPU \cite{FirefoxWebGPUSupport}. Thanks to the extensive conformance test suite \cite{WebGPUConformanceTestSuite}, it is more likely that the different implementations will be compatible with each other.

The main browser which supports WebGPU to date is Chrome. WebGPU has shipped to general use on desktops in May of 2023 \cite{ChromeWebGPUSupport}. Since January 2024, WebGPU is also supported on modern Android devices \cite{ChromeAndroidWebGPUSupport}.

This means that it's straightforward to use WebGPU on most modern devices with the notable exception of Apple iOS and iPadOS devices.

\subsubsection{Missing Features}

To date, WebGPU does not support some features that are common in modern rendering APIs.

\paragraph{Ray Tracing}

\glspl{API} such as Vulkan support hardware-accelerated ray tracing \cite{vulkanRayTracing}. This entails helpers for building common acceleration structures, such as \gls{BVH}, as well as ray querying functions to determine intersections. WebGPU does not yet support these features, but there are discussions ongoing to add extensions \cite{webGPURayTracing} as well as a demonstration implemented in a Dawn fork \cite{webGPURayTracingFork}.


\section{Future Work}

\subsection{Web Worker Support}

Web Workers are a web technology that allows running scripts off the main thread. This can be used to offload \gls{CPU}-heavy tasks to a separate thread to prevent blocking the main thread which is responsible for rendering the user interface.

\subsection{BVH Construction}

The current implementation builds the \gls{BVH} on the \gls{CPU} and transfers it to the \gls{GPU}. Corresponding research \cite{lauterbach2009GPUbvh} suggests that moving parts of the construction to the \gls{GPU} directly could improve performance. This would reduce the amount of data that needs to be transferred between the \gls{CPU} and the \gls{GPU}. The new \gls{GPGPU} capabilities of WebGPU further enable this approach.

\subsection{ReSTIR}
\subsection{SHaRC}

https://intro-to-restir.cwyman.org/