%%% Local Variables:
%%% mode: latex
%%% TeX-master: "../main"
%%% coding: utf-8
%%% End:
% !TEX TS-program = pdflatexmk
% !TEX encoding = UTF-8 Unicode
% !TEX root = ../main.tex

The main goal is to implement a web-based ray tracer addressing the problem of the large number of variants for CAD-based product visualization leveraging data close to the production CAD models.

\section{Scene Description}
Lights, Materials, Camera, Objects
\section{Implementation}
\subsection{Ray Tracing}
\subsubsection{Uber Shader}

The OpenPBR standard is based on an über shader approach. This differs from node-based approaches such as MaterialX in that it uses a fixed set of inputs which can be configured. This approach offers a good balance between flexibility and performance.

\subsubsection{RGB, Spectral}
\subsubsection{Memory}

Memory Alignment (GPU), Memory Management (GPU+CPU)

\subsection{Random Number Generator}

To simulate light transport, the path tracer needs to pick random directions. Generating a random direction vector, defined as $v = (x, y, z)$, can be achieved by picking three random numbers independent of one another. This is done using a random number generator (RNG). There are a variety of RNGs available which differ in characteristics. For domains such as cryptography, it is important that the RNG is unpredictable. However, for a ray tracer, it is more important to have high performance. Therefore, a pseudorandom number generator is used.

A number of suitable RNGs are available, including Mersenne Twister \cite{rngMersenneTwister}, Xorshifts such as the method described by Marsaglia \cite{marsaglia2003xorshift} and more.

However, the renderer uses PCG as the RNG due to its good performance and quality \cite{o2014pcg}.

\subsection{WebGPU}
\subsection{Integration}
\section{Use Case Scenarios}
\section{Performance}