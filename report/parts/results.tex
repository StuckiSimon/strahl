%%% Local Variables:
%%% mode: latex
%%% TeX-master: "../main"
%%% coding: utf-8
%%% End:
% !TEX TS-program = pdflatexmk
% !TEX encoding = UTF-8 Unicode
% !TEX root = ../main.tex

The main goal is to implement a web-based path tracer addressing the problem of the large number of variants for CAD-based product visualization leveraging data close to the production CAD models.

The result of this work consists of multiple parts. The report serves as the primary documentation of the work, but does not contain the different details. The library and detailed code documentation are published under the MIT license on GitHub. See \url{https://www.github.com/StuckiSimon/strahl}. In addition, a dedicated short-paper has been published for WEB3D '24: The 29th International ACM Conference on 3D Web Technology \cite{ownShortPaper}. The short-paper includes the main insights and results of this work.

This section contains references to the implementation of the path tracer. References are formatted in a consistent manner and can be searched for in the code. All relevant places are marked in code using the same reference. For example \coderef{ABC} refers to the code with the same comment.

\section{Scene Description}
Lights, Materials, Camera, Objects
\section{Implementation}
\subsection{Ray Tracing}
\subsubsection{Uber Shader}

The \gls{OpenPBR} standard is based on an uber shader approach. This differs from node-based approaches such as \gls{MaterialX} in that it uses a fixed set of inputs which can be configured. This approach offers a good balance between flexibility and performance.

\subsubsection{RGB, Spectral}
\subsubsection{Memory}

Memory Alignment (GPU), Memory Management (GPU+CPU)

\subsection{View Projection}

For many applications, especially photorealistic rendering, perspective projection is used. Based on the assessed use cases, the path tracer uses perspective projection only. See \coderef{VIEWPROJECTION} for implementation.

\subsection{Random Number Generator}

The path tracer uses PCG-RXS-M-XS variant as described by O’Neill \cite{o2014pcg} in combination with Xorshift as described by Marsaglia \cite{marsaglia2003xorshift}. See \coderef{RNG} for implementation.

\subsection{Intersection Testing}

For \gls{BVH} construction, well-established solutions for the web are available. The path tracer uses \texttt{three-mesh-bvh} \cite{threeMeshBvh}. This method builds the \gls{BVH} on the \gls{CPU}, the code for transfering the \gls{BVH} to the \gls{GPU} is in \coderef{BVH-TRANSFER}, intersection tests are implemented in \gls{WGSL}, see \coderef{BVH-TESTS}.

\subsection{Anti-Aliasing}

The implementation of the strategy indicated in \ref{sec:anti-aliasing} is implemented in \coderef{ALIASING}.


\subsection{WebGPU}
\subsection{Integration}
\section{Use Case Scenarios}
\section{Performance}